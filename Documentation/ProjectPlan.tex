%----------------------------------------------------------------------------------------
%	PACKAGES AND OTHER DOCUMENT CONFIGURATIONS
%----------------------------------------------------------------------------------------

\documentclass[paper=a4, fontsize=13pt]{scrartcl} % A4 paper and 11pt font size

\usepackage[utf8]{inputenc}
\usepackage[T1]{fontenc} % Use 8-bit encoding that has 256 glyphs
\usepackage{fourier} % Use the Adobe Utopia font for the document - comment this line to return to the LaTeX default
\usepackage[english]{babel} % English language/hyphenation
\usepackage{amsmath,amsfonts,amsthm} % Math packages

\usepackage{lipsum} % Used for inserting dummy 'Lorem ipsum' text into the template

\usepackage{sectsty} % Allows customizing section commands
\allsectionsfont{\centering \normalfont\scshape} % Make all sections centered, the default font and small caps

\usepackage{fancyhdr} % Custom headers and footers
\pagestyle{fancyplain} % Makes all pages in the document conform to the custom headers and footers
\fancyhead{} % No page header - if you want one, create it in the same way as the footers below
\fancyfoot[L]{} % Empty left footer
\fancyfoot[C]{} % Empty center footer
\fancyfoot[R]{\thepage} % Page numbering for right footer
\renewcommand{\headrulewidth}{0pt} % Remove header underlines
\renewcommand{\footrulewidth}{0pt} % Remove footer underlines
\setlength{\headheight}{13.6pt} % Customize the height of the header

\numberwithin{equation}{section} % Number equations within sections (i.e. 1.1, 1.2, 2.1, 2.2 instead of 1, 2, 3, 4)
\numberwithin{figure}{section} % Number figures within sections (i.e. 1.1, 1.2, 2.1, 2.2 instead of 1, 2, 3, 4)
\numberwithin{table}{section} % Number tables within sections (i.e. 1.1, 1.2, 2.1, 2.2 instead of 1, 2, 3, 4)

\setlength\parindent{0pt} % Removes all indentation from paragraphs - comment this line for an assignment with lots of text

%----------------------------------------------------------------------------------------
%	TITLE SECTION
%----------------------------------------------------------------------------------------

\newcommand{\horrule}[1]{\rule{\linewidth}{#1}} % Create horizontal rule command with 1 argument of height

\title{	
\normalfont \normalsize 
\textsc{Ruprecht-Karls-Universität Heidelberg, Institut für Informatik} \\ [25pt] % Your university, school and/or department name(s)
\horrule{0.5pt} \\[0.4cm] % Thin top horizontal rule
\huge Exercise 6: Project Proposal \\ DeepVision\\ % The assignment title
\horrule{2pt} \\[0.5cm] % Thick bottom horizontal rule
}

\author{Team: Max Schumacher, Oliver Schmitt} % Your name

\date{\normalsize\today} % Today's date or a custom date

\begin{document}

\maketitle

\begin{description}
\item{Contribution of the team members:} \\
\begin{itemize}
\item Oliver:
\begin{itemize}
\item Data preprocessing and implementation of a Data-Loader
\item Hyper-parameter tuning and training of the model
\item Evaluation of the results
\end{itemize}
\item Max:
\begin{itemize}
\item Reimplementing the YOLO network in python (model)
\item Tuning of a data sampler
\item Evaluation of the results
\end{itemize}
\end{itemize}
\item{Problem Definition:} \\
For our project, we want to deal with object detection in videos in real-time. This topic is interesting, since it is applicable to numerous real life scenarios e.g. autonomous driving cars and camera surveillance.
Since we want to apply object detection to videos our goal is to find a solution that is both, reasonably fast and accurate. For our approach we want to confidently predict 50 classes in real time (5 fps or more).
\item{Dataset:} \\
For the data, we want to use weights from a pretrained model that relies on ImageNet \cite{ImageNet} and afterwards train our model on the CitySpaces-dataset \cite{CitySpaces}, which contains semantic segmentation for 25.000 images. However, this dataset could be easily substituted for any other dataset containing labelled semantic segmentation for images.
\item{Approach:} \\
For our approach we want to use the YOLOv3 Model \cite{YOLO}. Therefore, we want to reimplement the model in python with a deep learning library like: PyTorch or TensorFlow. Since the model is currently implemented via DarkNet \cite{DarkNet} in C we want to transport the model to python while keeping acceptable performance; in terms of frame-rate. The result could either be a native implementation in python, or a python-wrapped version of the original implementation. Afterwards we want to load the weights of a pre-trained YOLO-model and use these to further train on the City-Spaces dataset, to see how the performance of the network improves when trained on unseen data. 
\item{Evaluation and expected Results:} \\
Firstly, we want to compare computation-time of our implementation to the original implementation(DarkNet). Although we expect a lower frame-rate, our implementation should keep performance that is suitable for real-time video processing. (5 fps or more) \\
Additionally, we want to compare accuracies, on the test set from the CitySpaces-dataset; before and after training the pre-trained model. Simultaneously, we analyze how the accuracy behaves on the ImageNet dataset. We expect a gain in accuracy concerning the CitySpaces-dataset and a loss of accuracy with the ImageNet-dataset. Therefore, we will also explore techniques of data-sampling that take training samples from both datasets to mitigate this problem.
\newpage
\item{Hardware:}
\begin{itemize}
\item For training we want to take advantage of the CUDA-Framework. Therefore, we plan on using a Nvidia gtx 1060 with 6GB of VRAM as compute unit
\item The rest of the system, we plan on using, consists of an i7 8600k and 16GB of RAM
\end{itemize}
\end{description} 

\begin{thebibliography}{9}

\bibitem{ImageNet}
  Stanford vision, Stanford University: ImageNet - \textit{http://www.image-net.org/}

\bibitem{CitySpaces}
https://www.cityscapes-dataset.com/

\bibitem{YOLO}
https://pjreddie.com/darknet/yolo/


\bibitem{DarkNet}
https://pjreddie.com/darknet/
\end{thebibliography}
\end{document}
